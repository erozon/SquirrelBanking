\section{Total Reproductive Output}
Christoph sends the following in an email: \\ \\

``\ldots wondering whether it is even necessary to try to tie everything back to Lyapunov exponents in Leslie matrices. 
Instead, I think all that is needed is to calculate the lifetime reproductive output of an individual (which follows from 
your graphs). This is an excellent measure of fitness. The only challenge is the comparison between individuals with different life histories and, 
in particular, different life expectancies. A reasonable approximation would be to raise the fitness of the shorter lived 
type to some power that depends on the ratio of the two life expectancies.'' \\ \\

Here is a first attempt: denote by $\int \text{LH}(a) \d a$ = RR the total expected number of offspring for an individual with life history
LH, and as a (terrible) shorthand, denote $\mathbb{E}$ its life expectency. Propose the following measure of fitness:

$$ f := \left( \int \text{ LH} (a) \d a \right)^{\frac{1}{\mathbb{E}}}.$$

This looks to be somewhat arbitrary, but it actually has some nice properties.

\begin{itemize}
    \item Not replacing oneself leads to fitness $< 1$ irrespective of the life expectancy, since $r^x < 1$ for all $r\in(0,1)$ and
        $x > 0$. 
    \item Low life expectancies with replacement rate $> 1$ are rewarded; if $r > 1$ and $\mathbb{E}$ is large, then we have something
        of the form $r^x$ where $x$ is large, which blows up pretty quickly. 
    \item Comparing a smaller/sooner with a larger/later life history yields exactly the ratio of life expectancies referenced in 
        Christoph's email:
        \begin{align*}
            f_{\text{SS}} = f_{\text{LL}} &\iff \left( \text{RR}_{\text{SS}} \right)^{\frac{1}{\mathbb{E}_{\text{SS}}}} =
                \left( \text{RR}_{\text{LL}} \right)^{\frac{1}{\mathbb{E}_{\text{LL}}}} \\
                &\iff \left( \text{RR}_{\text{SS}} \right)^{\frac{\mathbb{E}_{\text{LL}}}{\mathbb{E}_{\text{SS}}}} = \text{RR}_{\text{LL}}
        \end{align*}
\end{itemize}

